Bad news, just got notified that our MICCAI paper was rejected :(

===================================
Sarah E. Gerard
Ph.D. Candidate & Graduate Research Assistant
Biomedical Engineering
1440 Seamans Center
The University of Iowa
===================================
Stay hungry. Stay foolish.

________________________________________
From: msabuncu@nmr.mgh.harvard.edu <msabuncu@nmr.mgh.harvard.edu>
Sent: Friday, April 29, 2016 10:56 AM
To: Gerard, Sarah E
Cc: msabuncu@nmr.mgh.harvard.edu
Subject: MICCAI 2016 notification - #763

Dear Sarah Gerard

We regret to inform you that your submission
763 - Alpha Shapes for Lung Segmentation in the Presence of Large Tumors
was rejected in the early decision phase of the review process due to
insufficient support from the external reviewers. The anonymous
reviews of your paper are listed below and can also be seen in the
submission system.

We had a significant number of papers submitted to MICCAI 2016 and
competition for acceptance is particularly tough this year.

Please consider submitting a contribution to one of the satellite
workshops. The deadlines for most workshops are in early June 2016.

With best wishes,
Sandy Wells, Mert Sabuncu, Leo Joskowicz, and Gözde Ünal
MICCAI 2016 Program Chairs

-------------------------------------------

------------------------ Submission 763, Review 1 ------------------------

Title: Alpha Shapes for Lung Segmentation in the Presence of Large Tumors


Overall Recommendation

   Short of acceptable - Probably reject

Out of scope


Expertise

   Expert

Summary, Contributions and Significance

   The authors propose to use alpha shapes to initialize a graph-cut
   algorithm for the segmentation of the lungs in 3D CT data. Alpha shapes
   are a generalization of the convex hull, where alpha allows to tune
   between a tight shape (small alpha) and the convex hull (large alpha).
   The lung is first segmented using region growing; the alpha shape of that
   is computed and then smoothed; an existing graph cut method is applied
   using the smoothed alpha shape as the prior; the OR is taken between the
   graph cut result and the region growing result. The method is evaluated
   using 12 thoracic CT scans that include large tumors, using the mean
   surface distance and the dice overlap with manual segmentations as
   evaluation criteria. Compared to just region growing the results are
   improved.

Strengths

   The paper is clearly written and it is a nice idea to exploit alpha
   shapes as a way to remove large holes in the initial region growing
   segmentation. The topic is of interest to our community.

Weaknesses

   As both alpha shapes and graph cut lung segmentation with a shape prior
   exist, the contribution of the paper is in the combination of both. This
   is a nice idea, but not groundbreaking. The evaluation compares the
   proposed method only with the region growing result, while it would have
   been interesting to perform a more thorough comparison. Questions that
   arise are for example: What is the performance when the region growing
   method was chosen as a shape prior? What if simply the convex hull was
   chosen as a shape prior? How robust is the graph cut for changes in
   alpha, as perhaps quite similar end results are obtained for different
   alphas. Could the graph cut perform equally well with a worse
   initialization when the search profile length of the graph is increased?
   It therefore remains somewhat unclear what the exact contribution of
   alpha shapes are wrt this problem.

Constructive Feedback

   - Section 2.2 of the paper could be somewhat condensed. Section 2.3 could
   then benefit from some extra explanation of the graph cut and the shape
   prior.
   - As the lung is a large structure, the Dice overlap is less meaningful.
   Also the mean of the surface distances is perhaps less informative. It
   would therefore be of interest to include maximum errors (Hausdorf
   distance) or better yet a visualization of the distribution of the errors
   (box-plot).
   - As the mediastinum is indeed quite subjective to annotate you may
   consider adding results only showing the distance errors at the lung
   surface excluding this region. It could simply be defined by a
   ball-shaped region similar to what is done in the EMPIRE10 challenge.
   - Why is the data resampled to isotropic voxel volumes? Is there a
   limitation in the algorithm or in the implementation, or is there another
   reason for doing that?
   - Regarding the patient-specific selection of alpha (future work item),
   plotting alpha against the resulting alpha shape volume or number of
   edges may be informative here. Sudden changes indicate the inclusion of
   larger regions.


------------------------ Submission 763, Review 2 ------------------------

Title: Alpha Shapes for Lung Segmentation in the Presence of Large Tumors


Overall Recommendation

   Poor quality - Reject

Out of scope


Expertise

   Expert

Summary, Contributions and Significance

   The paper presents an approach for the segmentation of lungs in CT images
   in the presence of large tumors. The segmentation method consists of four
   steps: (1) initial threshold-based segmentation given in Ref. [3]; (2)
   generation of an alpha-shape from the segmentation (for given alpha); (3)
   graph-cut based segmentation refinement; (4) combination of initial
   segmentation and graph-cut result.
   The method was tested on twelve breath-hold CT images and shows
   significant improvements compared with the initial threshold-based
   segmentation.

Strengths

   The contribution of the paper is the application of alpha-shapes to lung
   segmentation.
   The paper is well written. The method is straightforward, easy to
   implement and improves the segmentation performance in the presence of
   tumors compared to threshold-based methods. The research is
   comprehensible and reproducible because all parameters, implementation
   hints and computation times are given in the paper.

Weaknesses

   The motivation of the paper is unclear. The introduction focus on 4D CT
   images, however, only 3D image information is used in the method and in
   the evaluation. The application to 4D data is not addressed.
   The contribution of the paper is the use of alpha-shapes in step (2) of
   the pipeline. All other steps, including the refinement based on the
   combination of graph-cuts and threshold based segmentation were presented
   before, see e.g.[1] for a nearly identical pipeline for 4D image data
   using 4D SSMs as central step.
   The motivation for using alpha-shapes remains nebulous. The obvious
   limitation of this approach -- finding the correct alpha value -- is
   addressed by an manual optimization on the test data: “We empirically
   determined α = 25 to give good results for all subjects in this
   study.”
   There is no evidence that an sufficient alpha can be found in other data
   sets, e.g. for tumors with large connections to the lung boundary. (see
   e.g. the tumor in Fig. 1 of the mentioned paper [1]).
   Further, the the algorithm needs a large computational effort due to the
   necessary Delaunay triangulation of a large number of points in 3D. This
   will possibly prevent the direct application to 4D images.

   [1] Wilms, Matthias, Jan Ehrhardt, and Heinz Handels. "A 4D statistical
   shape model for automated segmentation of lungs with large tumors."
   Medical Image Computing and Computer-Assisted Intervention–MICCAI 2012.
   Springer Berlin Heidelberg, 2012. 347-354.

Constructive Feedback

   The authors should address the application of this approach to 4D CT
   images as motivated in the introduction. Further, approaches for an
   automatic selection of the alpha value should be presented.
   I further miss a discussion what assumption are made to tumor size and
   position in order to ensure the successful application of this method.


------------------------ Submission 763, Review 3 ------------------------

Title: Alpha Shapes for Lung Segmentation in the Presence of Large Tumors


Overall Recommendation

   Poor quality - Reject

Out of scope


Expertise

   Expert

Summary, Contributions and Significance

   Authors presented an automatic method to segment lungs with
   large tumors in CT images using an initial intensity based segmentation
   followed by alpha shape construction and graph search.

Strengths

   1- Paper is written in a simple language
   2- The problem is clear
   3- The method is described in a simple way.

Weaknesses

   1-The introduction is short
   2- The literature review did not cover the state-of-art.
   3- There is no contributions in the proposed work. All steps are
   previously proposed by others.
   4- 12 subjects are very low number to test their approach.
   5- The number of subjects per each groups of datasets (with and without
   tumor) is not clear. Authors mentioned that they divided them 11+1 but
   the figure 4 didn't show this.
   6-There is no comparison with other methods.
   7- Discussion of the results are completely insufficient.
   8- There are many typos.

Constructive Feedback

   The only contribuition of this paper is the application of  others
   apporach on lung imaging, and  I think the paper is really not ready for
   a publication at MICCAI.
















