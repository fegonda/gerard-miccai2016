% This is LLNCS.DEM the demonstration file of
% the LaTeX macro package from Springer-Verlag
% for Lecture Notes in Computer Science,
% version 2.4 for LaTeX2e as of 16. April 2010
%
\documentclass{llncs}
%
\usepackage{makeidx}  % allows for indexgeneration
%
\usepackage{todonotes}
%

\begin{document}
%
\frontmatter          % for the preliminaries
%
\pagestyle{headings}  % switches on printing of running heads
\addtocmark{Alpha Shapes for Tumor Inclusion} % additional mark in the TOC
%

%
\tableofcontents
%
\listoftodos

\mainmatter              % start of the contributions
%
\todo[author=SG,inline]{Share document with Prof. Reinhardt once this is in at least a rough draft form}
\todo[author=SG,inline]{Change title and running title}
\title{Ultimate Best Total Lung Segmentation via Alpha Shapes in the presence of large volume tumors}
%
\titlerunning{Alpha Shape Lung Segmentation}  % abbreviated title (for running head)
%                                     also used for the TOC unless
%                                     \toctitle is used
%
\todo[author=All, inline]{\\citep command available for parenthesis citation with natbib}
\todo[author=SG, inline]{Set Authors}
\author{Sarah E. Gerard\inst{1} \and Hans J. Johnson\inst{1}
\and Joseph M. Reinhardt\inst{1}}
%
\authorrunning{Sarah E. Gerard et al.} % abbreviated author list (for running head)
%
%%%% list of authors for the TOC (use if author list has to be modified)
\tocauthor{Sarah E. Gerard, Hans J. Johnson, Joseph M. Reinahrdt}
%
\institute{University of Iowa, Iowa City IA 52402, USA,\\
\email{sarah-gerard@uiowa.edu}
}


\maketitle              % typeset the title of the contribution

\begin{abstract}
Lung segmentation is a critical initial step for radiation therapy interventions of lung cancer patients. However, automatic segmentation of lungs with large tumors is a challenging task due to the large variation of both lung and tumor shape between subjects. We present a method that utilizes alpha shapes to automatically segment lungs with large tumors in CT images.

\keywords{segmentation, alpha shapes, lung}
\end{abstract}
%
\section{Introduction}
%
Radiation therapy interventions of lung cancer patients involve acquiring a thoracic CT scan. Typically a respiratory correlated scan, or 4D CT, is also acquired prior to treatment. This produces a huge amount of data, requiring automatic computer-aided methods for further analysis. Accurate delineation of the lungs is a critical initial first step for treatment planning and lung function analysis using image registration.

Normal lungs have high contrast with the surrounding anatomy, making lung segmentation relatively trivial using thresholding based methods ~\cite{guo2008,hu2001}. However, in the presence of large tumors, conventional threshold based methods fail because tumors are excluded from the segmentation.

Several groups have used statistical shape models to segment pathological lungs in CT images ~\cite{sun2012,sofka2011}. These methods are able to include most pathology since the algorithm uses lung shape statistics rather than image intensity alone. However, building the model requires a large training set with annotated corresponding landmark points. Additionally, the training set is limited to a subset of possible lung shape variations and thus it could fail on cases that are not represented in the training set.

Segmentation by registration has also been used to segment pathological lungs ~\cite{sluimer2005,vanrikxoort2009}. An atlas database is used that contain a ground truth segmentation, and image registration is used to map the atlas segmentation to the test case. This method requires a time consuming registration.

Blah I feel like crap I'm going home.



%
\section{Methods}
%

%
\subsection{Initial Segmentation}
%
Describe Junfeng's segmentation here.
%
\subsection{Alpha Shape}
%
Delaunay triangulation, convex hull, alpha shapes
%
\subsection{Graph Search}
%
Optimal surface finding fun!
%
\section{Data Sets and Experimental Setup}
%
Somebody give me manual segmentations so I can show how awesome our method is.

Using the jupyter notebook \cite{PER-GRA:2007} rapid prototyping environmnet,  algorithms from ITK\cite{johnson2015itk} python wrapped in SimpleITK \cite{10.3389/fninf.2013.00045} in coordination with vtkDelaunay3D from VTK (www.vtk.org).

%
\section{Results}
%
Sarah's results are clearly best!!!!!!!!!!
%
\section{Summary}
%





\iffalse

\begin{figure}
\vspace{2.5cm}
\caption{This is the caption of the figure displaying a white eagle and
a white horse on a snow field}
\end{figure}


%
\paragraph{Notes and Comments.}


\begin{table}
\caption{This is the example table taken out of {\it The
\TeX{}book,} p.\,246}
\begin{center}
\begin{tabular}{r@{\quad}rl}
\hline
\multicolumn{1}{l}{\rule{0pt}{12pt}
                   Year}&\multicolumn{2}{l}{World population}\\[2pt]
\hline\rule{0pt}{12pt}
8000 B.C.  &     5,000,000& \\
  50 A.D.  &   200,000,000& \\
1650 A.D.  &   500,000,000& \\
1945 A.D.  & 2,300,000,000& \\
1980 A.D.  & 4,400,000,000& \\[2pt]
\hline
\end{tabular}
\end{center}
\end{table}
\begin{figure}
    \centering
    \includegraphics{}
    \caption{Caption}
    \label{fig:my_label}
\end{figure}

\fi
%
% ---- Bibliography ----
%
\bibliography{refs.bib}
\bibliographystyle{splncs03}







\clearpage
\addtocmark[2]{Author Index} % additional numbered TOC entry
\renewcommand{\indexname}{Author Index}
\printindex
\clearpage
\iffalse
\addtocmark[2]{Subject Index} % additional numbered TOC entry
\markboth{Subject Index}{Subject Index}
\renewcommand{\indexname}{Subject Index}
%                                                           clmomu01.ind
%-----------------------------------------------------------------------
% CLMoMu01 1.0: LaTeX style files for books
% Sample index file for User's guide
% (c) Springer-Verlag HD
%-----------------------------------------------------------------------
\begin{theindex}
\item Absorption\idxquad 327
\item Absorption of radiation \idxquad 289--292,\, 299,\,300
\item Actinides \idxquad 244
\item Aharonov-Bohm effect\idxquad 142--146
\item Angular momentum\idxquad 101--112
\subitem algebraic treatment\idxquad 391--396
\item Angular momentum addition\idxquad 185--193
\item Angular momentum commutation relations\idxquad 101
\item Angular momentum quantization\idxquad 9--10,\,104--106
\item Angular momentum states\idxquad 107,\,321,\,391--396
\item Antiquark\idxquad 83
\item $\alpha$-rays\idxquad 101--103
\item Atomic theory\idxquad 8--10,\,219--249,\,327
\item Average value\newline ({\it see also\/} Expectation value)
15--16,\,25,\,34,\,37,\,357
\indexspace
\item Baker-Hausdorff formula\idxquad 23
\item Balmer formula\idxquad 8
\item Balmer series\idxquad 125
\item Baryon\idxquad 220,\,224
\item Basis\idxquad 98
\item Basis system\idxquad 164,\,376
\item Bell inequality\idxquad 379--381,\,382
\item Bessel functions\idxquad 201,\,313,\,337
\subitem spherical\idxquad 304--306,\, 309,\, 313--314,\,322
\item Bound state\idxquad 73--74,\,78--79,\,116--118,\,202,\, 267,\,
273,\,306,\,348,\,351
\item Boundary conditions\idxquad 59,\, 70
\item Bra\idxquad 159
\item Breit-Wigner formula\idxquad 80,\,84,\,332
\item Brillouin-Wigner perturbation theory\idxquad 203
\indexspace
\item Cathode rays\idxquad 8
\item Causality\idxquad 357--359
\item Center-of-mass frame\idxquad 232,\,274,\,338
\item Central potential\idxquad 113--135,\,303--314
\item Centrifugal potential\idxquad 115--116,\,323
\item Characteristic function\idxquad 33
\item Clebsch-Gordan coefficients\idxquad 191--193
\item Cold emission\idxquad 88
\item Combination principle, Ritz's\idxquad 124
\item Commutation relations\idxquad 27,\,44,\,353,\,391
\item Commutator\idxquad 21--22,\,27,\,44,\,344
\item Compatibility of measurements\idxquad 99
\item Complete orthonormal set\idxquad 31,\,40,\,160,\,360
\item Complete orthonormal system, {\it see}\newline
Complete orthonormal set
\item Complete set of observables, {\it see\/} Complete
set of operators
\indexspace
\item Eigenfunction\idxquad 34,\,46,\,344--346
\subitem radial\idxquad 321
\subsubitem calculation\idxquad 322--324
\item EPR argument\idxquad 377--378
\item Exchange term\idxquad 228,\,231,\,237,\,241,\,268,\,272
\indexspace
\item $f$-sum rule\idxquad 302
\item Fermi energy\idxquad 223
\indexspace
\item H$^+_2$ molecule\idxquad 26
\item Half-life\idxquad 65
\item Holzwarth energies\idxquad 68
\end{theindex}

\fi
\end{document}
